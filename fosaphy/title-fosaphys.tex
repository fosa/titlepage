% Dies ist das Titelblatt Design der FOSAPHY.
% Die tikz-Grafik wurde inspiriert durch:
% http://tex.stackexchange.com/questions/44252/best-way-to-draw-a-quantum-harmonic-oscillator


\documentclass[a5paper,10pt,fleqn]{book}

\usepackage{mystyle}

\usepackage{trajan}

\usepackage{tikz}
\usepackage{tikz-3dplot}

\makeatletter
\newcommand*{\rom}[1]{\expandafter\@slowromancap\romannumeral #1@}
\makeatother


\begin{document}

\thispagestyle{empty}

\begin{Huge}
	\trjnfamily
	\noindent\rule{\textwidth}{2pt}
	\begin{center}
		\noindent
		%\textcolor{red}{SOLUTIO ON METHODIS}
		FORMELSAMMLUNG
	\end{center}
	\begin{center}
		\noindent
		%\textcolor{red}{MATHEMATICIS}	
		PHYSIK
	\end{center}
\end{Huge}

\vfill{}

\noindent\resizebox{\textwidth}{!}{
	\begin{tikzpicture}[background rectangle/.style={fill=gray},
		show background rectangle]

	\draw[white,->] (0,0) -- (0,4.5);

	\def\lzero{(3,0.4) cos (4,0.6) sin (5,0.8) cos (6,0.6) sin (7,0.4)};
	\fill[white,opacity=0.2] \lzero;
	\draw[white!50!white] \lzero node[right,white] {$\Psi_0(x)$};
	\draw[white] (-0.632,0.4) -- (2,0.4) node[right,white] {$E_0$};

	\def\lone{(3,1.5) cos (3.66,1.4) sin (4.33,1.3) cos (5,1.5) sin (5.66,1.7) cos (6.33,1.6) sin (7,1.5)};
	\fill[white,opacity=0.2] \lone;
	\draw[white!50!white] \lone node[right,white] {$\Psi_1(x)$};
	\draw[white] (-1.224,1.5) -- (2,1.5) node[right,white] {$E_1$};

	\def\ltwo{(3,2.6) cos (3.5,2.7) sin (4,2.8) cos (4.5,2.6) sin (5,2.4) cos (5.5,2.6) sin (6,2.8) cos (6.5,2.7) sin 			(7,2.6)};
	\fill[white,opacity=0.2] \ltwo;
	\draw[white!50!white] \ltwo node[right,white] {$\Psi_2(x)$};
	\draw[white] (-1.61,2.6) -- (2,2.6) node[right,white] {$E_2$};

	\draw[white] (0,0) parabola (2,4);
	\draw[white] (0,0) parabola (-2,4);
\end{tikzpicture}}

\vfill{}



\vfill{}

\begin{large}
	\begin{center}
		\normalfont
		\noindent
		Eine Zusammenarbeit von
			Daniel Winz,
			Ervin Mazlagi\'c,
			Mario Felder
			und 
			Marcel Holzmann
	\end{center}
\end{large}

\begin{large}	
	\noindent\rule{\textwidth}{2pt}
	\begin{center}
		\noindent
		% Version \rom{1} --- \today
		HS13
	\end{center}
\end{large}


\end{document}
