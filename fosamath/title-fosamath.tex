% Dies ist das Titelblatt Design der FOSAMATH.
% Die tikz-Grafik wurde inspiriert durch:
% http://www.texample.net/tikz/examples/fibonacci-spiral/

\documentclass[a5paper,10pt,fleqn]{book}

\usepackage{mystyle}

\usepackage{trajan}

\makeatletter
\newcommand*{\rom}[1]{\expandafter\@slowromancap\romannumeral #1@}
\makeatother

\begin{document}

\thispagestyle{empty}

\begin{Huge}
	\trjnfamily
	\noindent\rule{\textwidth}{2pt}
	\begin{center}
		\noindent
		FORMELSAMMLUNG
	\end{center}
	\begin{center}
		\noindent	
		MATHEMATIK
	\end{center}
\end{Huge}

\vfill{}

\noindent
\resizebox{\textwidth}{!}{
	\begin{tikzpicture}[
		background rectangle/.style={fill=gray}, 
		show background rectangle] 
  
	% Fibonacci Platzhalter
  	\newcounter{a}
  	\newcounter{b}
  	\newcounter{temp}

  	% Initialisierung der Fibonacci Platzhalter
  	\setcounter{a}{0}
  	\setcounter{b}{1}

  	% Startpunkt der Spirale festlegen
  	\coordinate (0) at (0,0);

	% Anzahl der Spiralumläufe definieren
	\foreach \i in {1,...,18}
 	{
		% Hole den "Namen" des letzten Punktes der Spirale
    		\pgfmathsetmacro{\lastpoint}{\i-1}

		% Berechne den Winkel für die Drehung der Spirale
    		\pgfmathsetmacro{\startangle}{mod(\i-1,4) * 90}

		% Zeichne die Spirale und speichere den Endpunkt
    		\draw[white] (\lastpoint) arc 
      		(\startangle : \startangle + 90 : \value{b}/10.0pt) coordinate (\i);

   		% Berechne die nächste Fibonacci Zahl
    		\setcounter{temp}{\value{b}}
    		\addtocounter{b}{\value{a}}
    		\setcounter{a}{\value{temp}}
 	}

 	% Teilrahmen der Spirale zeichnen
 	\foreach \i in {1,3,...,17}
 	{
   		\pgfmathsetmacro{\lastpoint}{\i-1}
   		\draw[white] (\lastpoint) -| (\i);
 	}

 	\foreach \i in {2,4,...,16}
 	{
   		\pgfmathsetmacro{\lastpoint}{\i-1}
   		\draw[white] (\lastpoint) |- (\i);
 	}

 	\draw[white] (17) -- (17 |- 18);
 
 	% Definition der Fibonacci Zahlen platzieren
 	\node(eq) at ($(18) + (2.5,1)$) 
   		[white,text width = 2cm,font=\fontsize{12}{12}\selectfont] {
   	\begin{displaymath}
     		f(n) = \left\{
       		\begin{array}{lr}
         		0 & n = 0\\
         		1 & n = 1\\
         		f(n-1) + f(n-2) & n \geq 2\\
      		\end{array}
     		\right.
   	\end{displaymath}
  	};
\end{tikzpicture}}

\vfill{}

\begin{large}
	\begin{center}
		\normalfont
		\noindent
		Eine Zusammenarbeit von
			Daniel Winz,
			Ervin Mazlagi\'c,
			Adrian Imboden,
			Philipp Langer,
			Josef Häfliger,
			und
			Marcel Holzmann
	\end{center}
\end{large}

\begin{large}	
	\noindent\rule{\textwidth}{2pt}
	\begin{center}
		\noindent
		Version \rom{4} --- \today
	\end{center}
\end{large}


\end{document}
