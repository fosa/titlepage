% Dies ist das Titelblatt Design der FOSAMATH.
% Die tikz-Grafik wurde inspiriert durch:
% http://www.texample.net/tikz/examples/perpendicular-bissector/


\documentclass[a5paper,10pt,fleqn]{book}

\usepackage{mystyle}

\usepackage{trajan}

\makeatletter
\newcommand*{\rom}[1]{\expandafter\@slowromancap\romannumeral #1@}
\makeatother


\begin{document}

\thispagestyle{empty}

\begin{Huge}
	\trjnfamily
	\noindent\rule{\textwidth}{2pt}
	\begin{center}
		\noindent
		%\textcolor{red}{SOLUTIO ON METHODIS}
		FORMELSAMMLUNG
	\end{center}
	\begin{center}
		\noindent
		%\textcolor{red}{MATHEMATICIS}	
		ELEKTROTECHNIK
	\end{center}
\end{Huge}

\vfill{}

\newcounter{index}
\setcounter{index}{0}

\noindent\resizebox{\textwidth}{!}{
\begin{tikzpicture}[
	background rectangle/.style={fill=gray},
	show background rectangle,
        scale=1.0,
        MyPoints/.style={draw=white,fill=white,thick},
        Segments/.style={draw=white,dotted},
        MyCircles/.style={white,thin}
        ]
    % Warning : all this is an artisanal way of computing points
    % on the perpendicular bissector of [AB]
    % It could very well be achieved with more powerfull tools...
    % (package tkz-2d, for example)
    \clip (-2,-2) rectangle (7,4);
    \draw[color=white,step=1.0,dotted] (-2.1,-2.1) grid (6.1,7.1);
    \draw[white,->] (-2,0)--(6.5,0) node[right]{$x$};
    \draw[white,->] (0,-2)--(0,3.5) node[above]{$y$};

    % Feel free to change here coordinates of points A and B
    \pgfmathparse{-1}		\let\Xa\pgfmathresult
    \pgfmathparse{-1}		\let\Ya\pgfmathresult
    \coordinate (A) at (\Xa,\Ya);
    \pgfmathparse{5}		\let\Xb\pgfmathresult
    \pgfmathparse{3}		\let\Yb\pgfmathresult
    \coordinate (B) at (\Xb,\Yb);
    
    % Let I be the midpoint of [AB]
    \pgfmathparse{(\Xb+\Xa)/2} \let\XI\pgfmathresult
    \pgfmathparse{(\Yb+\Ya)/2} \let\YI\pgfmathresult
    \coordinate (I) at (\XI,\YI);	
            
    \draw[white,thick] (A)--(B);
    
    % deltaX and deltaY are coordinates of vector AB
    \pgfmathparse{\Yb-\Ya} \let\deltaY\pgfmathresult
    \pgfmathparse{\Xb-\Xa} \let\deltaX\pgfmathresult
    
    % NormeddeltaX and NormeddeltaY are the normalized values of these coordinates
    \pgfmathparse{sqrt(\deltaX*\deltaX+\deltaY*\deltaY)} \let\r\pgfmathresult
    \pgfmathparse{\deltaX/\r} \let\NormeddeltaX\pgfmathresult
    \pgfmathparse{\deltaY/\r} \let\NormeddeltaY\pgfmathresult

    % R is a point on the perpendicular bissector of [AB],
    % far away from the midpoint...
    \pgfmathparse{\YI-10.0*\NormeddeltaX} \let\YR\pgfmathresult
    \pgfmathparse{\XI+10.0*\NormeddeltaY} \let\XR\pgfmathresult
    
    % S is the image of R by the symmetry of axis AB
    \pgfmathparse{2*\YI-\YR} \let\YS\pgfmathresult
    \pgfmathparse{2*\XI-\XR} \let\XS\pgfmathresult
    \coordinate (R) at (\XR,\YR);
    \coordinate (S) at (\XS,\YS);
    \draw[white] (R)--(S);
    
    \foreach \i in {-2,-1,...,2}{
        \ifthenelse{\equal{\i}{0}}% Do not redraw the segment [AB]
            {}%
            {%
                \stepcounter{index}
                % P(i) is a variable point on the perpendicular bissector.
                % The distance between P(i) and P(i+1) is equal to 1
                \pgfmathparse{\YI-\i*\NormeddeltaX} \let\YP\pgfmathresult
                \pgfmathparse{\XI+\i*\NormeddeltaY} \let\XP\pgfmathresult
                \coordinate (P) at (\XP,\YP);
                \pgfmathparse{sqrt((\XP-\Xa)*(\XP-\Xa)+(\YP-\Ya)*(\YP-\Ya))}
                \let\radius\pgfmathresult
                            
                \draw[white,MyCircles] (P) circle ({\radius});
                \draw[white,Segments] (P)--(A);
                \draw[white,Segments] (P)--(B);
                \fill[white,MyPoints] (P) circle (0.8mm) node[right]{$P_{\theindex}$};
            }%
        };
        
    \fill[white,MyPoints] (A) circle (0.8mm) node[left]{$A$};
    \fill[white,MyPoints] (B) circle (0.8mm) node[right]{$B$};
    \fill[white,MyPoints] (I) circle (0.8mm) node[right]{$I$};
\end{tikzpicture}}

\vfill{}

\begin{large}
	\begin{center}
		\normalfont
		\noindent
		Eine Zusammenarbeit von
			Daniel Winz und
			Ervin Mazlagi\'c
	\end{center}
\end{large}

\begin{large}	
	\noindent\rule{\textwidth}{2pt}
	\begin{center}
		\noindent
		Version \rom{3} --- \today
	\end{center}
\end{large}


\end{document}
