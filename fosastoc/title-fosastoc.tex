% Dies ist das Titelblatt Design der FOSASTOC.
% Die tikz-Grafik wurde inspiriert durch:
% http://www.texample.net/tikz/examples/gnuplot-basics/

\documentclass[a5paper,10pt,fleqn]{book}

\usepackage{mystyle}

\usepackage{trajan}

\makeatletter
\newcommand*{\rom}[1]{\expandafter\@slowromancap\romannumeral #1@}
\makeatother

\begin{document}

\thispagestyle{empty}

\begin{Huge}
	\trjnfamily
	\noindent\rule{\textwidth}{2pt}
	\begin{center}
		\noindent
		FORMELSAMMLUNG
	\end{center}
	\begin{center}
		\noindent	
		STOCHASTIK
	\end{center}
\end{Huge}

\vfill{}

\noindent
\resizebox{\textwidth}{!}{
	\begin{tikzpicture}[
		background rectangle/.style={fill=gray}, 
		show background rectangle, domain=-4:4, samples=200] 
	
	\draw[dotted, color=white] (-4, 0) grid (4, 4.25);
	\draw[color=white, ->] (-4, 0) -- (4, 0) node[right] {$x$};
	\draw[color=white, ->] (0, 0) -- (0, 4.25) node[right] {$P$};

	\draw[color=white] plot[id=x] 
	function{7*(1/(2*pi*0.5)**0.5)*exp(-1.1*(x)**2)} ;

	\draw[color=white] plot[id=x] 
	function{8*(1/(2*pi*1)**0.5)*exp(-1*(x)**2)} ;

	\draw[color=white] plot[id=x] 
	function{8*(1/(2*pi*1.5)**0.5)*exp(-0.9*(x)**2)} ;

	\draw[color=white] plot[id=x] 
	function{8*(1/(2*pi*2)**0.5)*exp(-0.8*(x)**2)} ;

	\draw[color=white] plot[id=x] 
	function{8*(1/(2*pi*2.5)**0.5)*exp(-0.7*(x)**2)} ;

	\draw[color=white] plot[id=x] 
	function{8*(1/(2*pi*3)**0.5)*exp(-0.6*(x)**2)} ;

	\draw[color=white] plot[id=x] 
	function{8*(1/(2*pi*3.5)**0.5)*exp(-0.5*(x)**2)} ;

	\draw[color=white] plot[id=x] 
	function{8*(1/(2*pi*4.0)**0.5)*exp(-0.4*(x)**2)} ;

	\draw[color=white] plot[id=x] 
	function{8*(1/(2*pi*4.5)**0.5)*exp(-0.3*(x)**2)} ;

	\draw[color=white] plot[id=x] 
	function{8*(1/(2*pi*5.0)**0.5)*exp(-0.2*(x)**2)} ;


\end{tikzpicture}}

\vfill{}

\begin{large}
	\begin{center}
		\normalfont
		\noindent
		Eine Zusammenarbeit von
			Daniel Winz 
			und
			Ervin Mazlagi\'c
	\end{center}
\end{large}

\begin{large}	
	\noindent\rule{\textwidth}{2pt}
	\begin{center}
		\noindent
		% Version \rom{4} --- \today
		HS13
	\end{center}
\end{large}


\end{document}
